\documentclass[
12pt, % The default document font size, options: 10pt, 11pt, 12pt
% oneside, % Two side (alternating margins) for binding by default, uncomment to switch to one side
english, 
onehalfspacing, % Single line spacing, alternatives: onehalfspacing or doublespacing
% draft, % Uncomment to enable draft mode (no pictures, no links, overfull hboxes indicated)
%nolistspacing, % If the document is onehalfspacing or doublespacing, uncomment this to set spacing in lists to single
%liststotoc, % Uncomment to add the list of figures/tables/etc to the table of contents
%toctotoc, % Uncomment to add the main table of contents to the table of contents
%parskip, % Uncomment to add space between paragraphs
%nohyperref, % Uncomment to not load the hyperref package
headsepline, % Uncomment to get a line under the header
%chapterinoneline, % Uncomment to place the chapter title next to the number on one line
%consistentlayout, % Uncomment to change the layout of the declaration, abstract and acknowledgements pages to match the default layout
]{MastersDoctoralThesis} % The class file specifying the document structure

\usepackage[utf8]{inputenc}
\usepackage[T1]{fontenc}
\usepackage{scrextend}
%\changefontsizes[14pt]{14pt}

% To include the City Pings table
% which was printed as a PDF via Google Sheets
% \usepackage{pdfpages}

\usepackage{palatino}
\usepackage{url}

% https://tex.stackexchange.com/questions/54055/using-lower-case-roman-numerals-in-enumerate-lists
\usepackage{enumerate}

% \usepackage{natbib}
% \bibliographystyle{unsrtnat}
\usepackage[backend=bibtex,citestyle=numeric,natbib=true,sorting=none]{biblatex}
\addbibresource{References.bib}

\usepackage[autostyle=true]{csquotes}

\usepackage{lipsum}
\usepackage{xargs}

% Leave todo comments in the PDF!
\usepackage[colorinlistoftodos,prependcaption,textsize=tiny]{todonotes}
\newcommandx{\bug}[2][1=]{\todo[linecolor=red,backgroundcolor=red!25,bordercolor=red,#1]{#2}}
\newcommandx{\info}[2][1=]{\todo[linecolor=green,backgroundcolor=green!25,bordercolor=green,#1]{#2}}
\newcommandx{\fixme}[2][1=]{\todo[linecolor=blue,backgroundcolor=blue!25,bordercolor=blue,#1]{#2}}

% https://www.overleaf.com/learn/latex/Code_Highlighting_with_minted
\usepackage{minted}
\setminted{fontsize=\footnotesize,baselinestretch=1,linenos,frame=lines,framesep=10pt}

% Figures directory is located in the root
\graphicspath{ {./figures/} }

\usepackage{titlesec}

% https://tex.stackexchange.com/questions/45817/theorem-definition-lemma-problem-numbering
% \usepackage{amsthm}
% \theoremstyle{definition}
% \newtheorem{defn}[thm]{Definition}

% https://tex.stackexchange.com/questions/89354/forcing-math-mode-to-be-on-the-same-line
% \binoppenalty=3000
% \relpenalty=3000

\usepackage{fixltx2e}

%----------------------------------------------------------------------------------------
%	MARGIN SETTINGS
%----------------------------------------------------------------------------------------

\geometry{
	paper=a4paper, % Change to letterpaper for US letter
	inner=2.5cm, % Inner margin
	outer=3.8cm, % Outer margin
	bindingoffset=.5cm, % Binding offset
	top=1.5cm, % Top margin
	bottom=1.5cm, % Bottom margin
	%showframe, % Uncomment to show how the type block is set on the page
}

%----------------------------------------------------------------------------------------
%	THESIS INFORMATION
%----------------------------------------------------------------------------------------

% \thesistitle{Fast Stabilization of PoW Blockchains}
\thesistitle{Improving security \& performance of PoW Blockchains using Anchors}
\supervisor{Prof. Vinay J. Ribeiro}
\degree{Master of Technology}
\author{Shadab Zafar}
\addresses{}

\subject{Blockchain}
\keywords{}

\university{IIT Delhi}
\department{Department of Computer Science and Engineering}
\faculty{Prof. Vinay J. Ribeiro}

\begin{document}

\frontmatter

\pagestyle{plain}

%----------------------------------------------------------------------------------------
%	TITLE PAGE
%----------------------------------------------------------------------------------------

\begin{titlepage}
\begin{center}
\includegraphics[width=30mm,height=30mm]{iitd-logo-large}
\\
\vspace*{.06\textheight}
{\scshape\LARGE \univname\par}\vspace{1.5cm}
\textsc{\Large Masters Thesis}\\[0.5cm]

\HRule \\[0.4cm]
{\huge \bfseries \ttitle\par}\vspace{0.4cm}
\HRule \\[1.5cm]

\begin{minipage}[t]{0.4\textwidth}
\begin{flushleft} \large
\emph{Author:}\\
\authorname
\end{flushleft}
\end{minipage}
\begin{minipage}[t]{0.4\textwidth}
\begin{flushright} \large
\emph{Supervisor:} \\
\supname
\end{flushright}
\end{minipage}\\[3cm]

\vfill

\large \textit{
	A thesis submitted in partial fulfillment of the requirements\\
	for the degree of \degreename
}\\[0.3cm]
	\textit{in the}\\[0.4cm]
	\deptname\\[1cm]

	{\large \monthyeardate\today}

\end{center}
\end{titlepage}

\makecertificate

%----------------------------------------------------------------------------------------
%	ABSTRACT PAGE
%----------------------------------------------------------------------------------------

\begin{abstract}
\addchaptertocentry{\abstractname}

After their first public use in Bitcoin, Blockchains have taken the world by storm.
Even though research has been done on alternative forms of consensus schemes and new algorithms have been proposed, Proof-of-Work (PoW) is still the most widely used approach.
This popularity has also shed light on some of the drawbacks that the current PoW Blockchains suffer from.
Since PoW consensus generates blocks at random time instants, the weight of the chain increases in an unsteady manner
which leads to forks, increased chance of double-spend and selfish mining attacks.
It also results in systems with low performance as the confirmation times increase and throughput decreases.

In this thesis, we present a new signalling mechanism (called Anchors) which aid in improving the security and performance of PoW blockchains, while adding minimal overhead to the system.
We implement Anchors in the reference implementation of Bitcoin (the most widely used PoW Blockchain) as a proof of concept.
We also create a testbed that allows running Bitcoin experiments and use it to evaluate our implementation of Anchors on a network of 114 Bitcoin nodes. 
Comparing the results with unmodified Bitcoin we find that anchors propagate much faster than blocks, aiding in quicker fork resolution times (and even preveing some forks from occurring), without adversely impacting block propagation times.

\end{abstract}

%----------------------------------------------------------------------------------------
%	ACKNOWLEDGEMENTS
%----------------------------------------------------------------------------------------

\begin{acknowledgements}
\addchaptertocentry{\acknowledgementname}
First, I'd like to thank \supname, without whose guidance and supervision this work would not have been possible.
Next, I'd like to thank my co-authors and lab-mates Ovia Seshadri and Aditya Kumar for discussing the finer details of the projects and helping me whenever I was stuck.

This work would've moved considerably slowly had it not been for the helpful folks in the Bitcoin community (Slack, IRC, StackOverflow, Reddit etc.)
I've received help from many people, but I'd specially like to thank Gregory Maxwell and Luke Dashjr for taking time out of their lives and helping me figure out the peculiarities of the Bitcoin Core client.
\end{acknowledgements}

%----------------------------------------------------------------------------------------
%	LIST OF CONTENTS/FIGURES/TABLES PAGES
%----------------------------------------------------------------------------------------

\tableofcontents

\listoffigures
\listoflistings

% \listoftables

%----------------------------------------------------------------------------------------
%	THESIS CONTENT - CHAPTERS
%----------------------------------------------------------------------------------------

\mainmatter

\pagestyle{thesis}

% https://tex.stackexchange.com/questions/59726/change-size-of-section-subsection-subsubsection-paragraph-and-subparagraph-ti

\chapter{Introduction} \label{ch-intro}

In this chapter we introduce the preliminaries:
Section \ref{intro-blockchain} begins with the definition of a blockchain and explains key terms adjacent to it.
Section \ref{intro-bitcoin} describes Bitcoin - the first public PoW blockchain system and its reference implementation - Bitcoin Core. 
Section \ref{intro-problems} lists some of the shortcomings of current PoW blockchains. 

%----------------------------------------------------------------------------------------
%	SECTION 1 - blockchain Systems
%----------------------------------------------------------------------------------------

\section{Blockchain Technology} \label{intro-blockchain}

% Many use Bitcoin as the starting point and explain blockchains by its first public use - as a cryptocurrency,

Blockchains are usually explained by describing Bitcoin, a cryptocurrency which was its first and most popular public application. 
However, blockchain technology now encompasses a wider scope and are much more general purpose than Bitcoin would suggest. 
Other systems currently being researched which include blockchains in their implementation include voting platforms, supply chain management, smart contracts, etc.

% Blockchains are usually explained by describing its first public application - a cryptocurrency (Bitcoin), but this does not capture other systems that are also categorized as blockchains. Such systems include applications of blockchain technology beyond cryptocurrency like in voting platforms, supply chain management, healthcare etc.

There exists no formal definition of a blockchain which is widely accepted; with multiple definitions being used in literature, each with slightly different phrasing. 

The simplest definition comes from Narayanan et. al. \cite{bitcoinBook}: 

\textit{
    A blockchain is defined as a linked list data structure, that uses hash sums over its elements as pointers to the respective elements.
}

In this view, a blockchain is a data structure where blocks (containers for storing records) are linked into a chain with the use of cryptographic hashes, and each new block stores a reference to its parent. 
% \info{need a pic here?}

Another definition comes from a technical committee formed by the International Standards Organization (ISO) to standardize blockchain technology \cite{isotc307} which define blockchain as:
\textit{
    a shared, immutable ledger that can record transactions across different industries, [...] 
    It is a digital platform that records and verifies transactions in a transparent and secure way, removing the need for middlemen and increasing trust through its highly transparent nature.
} 

Some keywords from the above definition require further explanation: 

\begin{itemize}
    \item \textit{shared} - a blockchain is distributed among a set of nodes which participate in its maintainance and are connected in a decentralized manner, so  there is no central party above others and all the nodes are equal in their capabilities.

    \item \textit{immutable} - In the context of blockchains, immutability refers to the inability to change data which is that has been published onto the chain.

    \item \textit{transactions}  - A transaction is generally considered to be the basic unit of data stored on a blockchain. The contents of a transaction can change, depending on the application of the blockchain and the context in which it is being used.
    % Across different industries the transactions may take different forms, but 

    \item \textit{transparent} - Transparency here refers to the fact that a blockchain and consequently its transactions can be reviewed and verified in their entirety by any participating node. 
    In Bitcoin's case, since the blockchain is public, any one can trace the history of any unit of bitcoin.

    \item \textit{middlemen \& trust} - Due to blockchains' decentralised nature, there is no need for middlemen (typically referred to as "trusted third party") to maintain and verify their correctness. 
    The removal of middlemen, coupled with transparency also ensures that there is no need for a node to trust any other node in order to verify the blockchain.
    
\end{itemize}

A blockchain can also be thought of as a distributed system, where the goal of participating nodes is to arrive at a consensus on the ordering of transactions in the chain. In this view, blockchains can be categorized into two broad types: 

\begin{itemize}
    \item \textbf{Permissionless: } The key property of this type of blockchain is that the set of nodes that take part in the consensus process (over the state of chain) is not known beforehand so anyone can join the network and participate in maintainance of the chain. These blockchains typically use a consensus algorithms such as Proof-of-Work, Proof-of-Stake etc.
    
    \item \textbf{Permissioned: } Contrary to the above, in this type of blockchain only a previously-known set of nodes is allowed to take part in the network. Furthermore, all the nodes do not have equal capabilities so for eg. only a restricted set of nodes might have the right to create or validate transactions etc. These blockchains typically use the PBFT (Pracrtical Byzantine Fault Tolerance) algorithm for consensus.
\end{itemize}

% In these blockchains, each miner tries to solve a computational puzzle and the first one to do so broadcasts its block which gets appended to the chain.

%----------------------------------------------------------------------------------------
\paragraph{Note: }
In this Thesis, we restrict ourselves to Blockchains that use Proof-of-Work as their consensus algorithm. 
Bitcoin, explained in the next section, is the largest publically deployed blockchain and uses PoW.

\section{Bitcoin} \label{intro-bitcoin}

Bitcoin is a digital currency (also called cryptocurrency) that was first proposed by the pseudonymous \footnote{It is unknown whether Satoshi is a person or a group of persons. Multiple people have tried to present "proof" that they're indeed Satoshi but there is still doubt.} "Satoshi Nakamoto". 
Satoshi self-published the Bitcoin whitepaper in 2008 \cite{bitcoinOriginal} and soon after, on January 3rd, 2009, the genesis block of the main Bitcoin blockchain \footnote{Satoshi seemed to prefer "block chain" over "blockchain" as shown by their comment in the original source code. \cite{satoshiComment}} was created.

Although the technical primitives (cryptographic hash functions, asymmetric cryptography) have existed for a while, 
Bitcoin was the first concept to combine these building blocks with an incentive system (mining rewards), thereby creating the first distributed cryptocurrency and it is by far the most successful one in terms of market capitalization reaching a peak value of 320 billion USD in Nov 2017 \cite{btcMarketCap}. 

% Addresses, transactions, and blocks are the three basic data structures used in Bitcoin. 

%----------------------------------------------------------------------------------------

\subsection{Proof-of-Work Consensus} \label{btc-pow}

At the heart of Bitcoin lies it's ingenious proof-of-work based consensus scheme (also called "Nakamoto Consensus").
A proof-of-work (PoW) in general enables a prover to provide evidence to a verifier that the prover has invested computational resources (CPU, memory etc.) into a task.
The idea of performing and providing a PoW for security reasons was developed and refined by Hal Finney, Adam Back and others \cite{hashcash, finneyRPoW}

In Bitcoin, a new block is only accepted to the blockchain if it contains a valid PoW.
Each new block points to an older one and can be seen as adding weight to that chain.
Honest nodes agree that at any point in time only the longest blockchain is considered valid. 
In practice, this is implemented as the "heaviest chain rule" - the chain having the most weight as it has been the hardest to compute.

PoW as used in Bitcoin is based on a partial pre-image attack on the cryptographic hash function SHA-256. 
This process can be seen as trying to solve a computational puzzle to find a block header (and an extra nonce value) that satisfy the relation: 
% \textit{ SHA\-256(SHA\-256(block\_header)) \leq T }
\mathchardef\mhyphen="2D

$$ SHA{\text -}256(SHA{\text -}256(block\_header)) \leq T $$

where T is the target space of valid solutions. 
% The higher the number T, the lower the number of possible solutions, resulting in the PoW being harder to find. 

In practice this may mean something like 'all blocks whose hashes begin with 69 zeroes are considered valid'.

% This process can be described as a “random leader election” where a leader is allowed to propose a new block and implicitly agrees on all blocks before that by appending its new block to the end of the respective blockchain

Proof-of-Work is based on the idea of 1-CPU-1-Vote and acts as a defense mechanism against "Sybil attacks" - where an attacker might create fake identities and gain a large influence on the network. \cite{sybil}

\subsection{Mining} \label{btc-mining}

Mining is the process of looking for a valid Block by performing PoW and reaching consensus on the current state of the blockchain. 
The nodes that are actively involved in searching for these solutions called \textit{miners}. 
They are rewarded with units of the "mined" cryptocurrency (bitcoins) as a compensation for investing computational power into the overall security of the cryptocurrency. 
Bitcoin is a prime example of \textit{permissionless} system, where anybody is allowed to connect to Bitcoin's peer-to-peer network and participate in the maintainance of the chain.
Miners can join or leave the network at any time, increasing or lowering the mining power. 
As the mining power changes, the hardness of the puzzle needs to be adjusted to ensure that new blocks are generated at regular intervals (which is 10 mintes for Bitcoin). 

While miners compete with each other for the mining reward, the Bitcoin network also has other nodes like a \textit{full node} - which might not be mining, but still contributes by forwarding the messages, storing a full copy of the blockchain \footnote{The current chain over 200 GBs in size \cite{bitcoinDownload}.} and participating in the consensus protocol.
% Nodes (in the Bitcoin network) that take part in generation of new blocks are called \textit{Miners}. 
% They compete with other miners who are also looking for valid blocks and 

%----------------------------------------------------------------------------------------

\newpage
\subsection{Bitcoin Core} \label{btc-arch}

The reference implementation of Bitcoin, known as Bitcoin Core \cite{bitcoinGithub}, was first written in C++ by Staoshi Nakamoto and released to the public on 9 Januray, 2009 \cite{bitcoinCoreWiki}. The development was later picked up by a community of developers, lead by Wladimir J. van der Laan (as of 2019), who continue to maintain the code on GitHub (a site that hosts open-source code repositories). 

% Mention BIPs?

% We used the stable version (v0.16 released on 26-02-2018) of Bitcoin Core that was available to us before we started working on this project. For reference, the latest stable release of Bitcoin is v0.18 and was released on 18-05-2019. 

Most of Bitcoin Core is written in C++ and is available for all major operating systems like Windows, macOS and Linux \cite{bitcoinDownload}. 

Bitcoin Core includes four clients, three of which are command line-based: 

\begin{itemize}
    \item \textit{bitcoind} - also called "bitcoin daemon" - is the main application that runs in background as a Bitcoin full node. It also contains a wallet implementation which can fully verify payments.

    \item \textit{bitcoin-cli} - once the daemon is running in background, this application acts as an interface for controlling the daemon's behaviour. It works by sending appropriate RPC commands to the daemon.

    \item \textit{bitcoin-tx} - this application provides a low-level interface for creating transactions (using the "raw transactions API" discussed in section \ref{impl-txns-raw})

    \item \textit{bitcoin-qt} - this GUI based application acts as a user-friendly frontend for the bitcoin daemon. The application has been written using the Qt UI Framework \cite{cppqt}.
\end{itemize}

The source repository also includes functional and unit-tests for the Core source code.

We now give some details on the design of Bitcoin Core:

\subsubsection{Modes of Operation} \label{btc-modes}

The Bitcoin daemon can be run in three modes which differ in the blockchain they maintain and the consensus parameters they use:

\begin{itemize}
    \item \textbf{Mainnet} - This is the default mode of operation, where the daemon connects to other nodes in the network and downloads the main Bitcoin blockchain acting as a full node by participating in the consensus protocol - verifying blocks and transactions.

    \item \textbf{Testnet} - The testnet is an alternative Bitcoin block chain, which is used for testing. 
    Testnet coins are separate and distinct from actual bitcoins, and are never supposed to have any value. 
    This allows application developers run experiments without having to use real bitcoins.

    \item \textbf{Regtest} - The regression test mode runs a local-only daemon which does not connect to any other node by default. 
    Furthermore, the consensus parameters for this mode are set such that blocks can be generated instantly (skipping any proof-of-work).
    This is meant for Bitcoin developers to test their code locally.
\end{itemize}

Testnet and Regtest modes are run by passing \textit{-testnet} or \textit{-regtest} parameters to the bitcoin daemon.
In our experimentation testbed, we used the regtest mode of Bitcoin since we wanted to avoid performing proof-of-work computation and run a private network of nodes.

\subsubsection{Protocol Interfaces} \label{btc-interfaces}

A running Bitcoin daemon has support for three protocols, each of which serves a distinct purpose and are exposed on a different TCP port as depicted in Figure \ref{fig-btc-arch}.

\begin{itemize}
    \item \textbf{Bitcoin Protocol} - This is the protocol that a Bitcoin node uses to send or receive data from other nodes in the Bitcoin network. It is a custom networking protocol built upon TCP, having its own message structures and types. \cite{bitcoinProtocol}
    Some messages of the protocol are: \textit{block}, \textit{tx}, \textit{addr} - each of which contain objects of a specific type.

    \item \textbf{REST API} - For clients that want to query the chain information, Bitcoin provides an unauthenticated REST API based on HTTP. \cite{btcREST} 
    The API allows using binary, hexadecimal and JSON (which is human readable) formats for accessing the data.
    This API is read-only, in that it can be used to read data like blocks and their contents, but can not be used to modify the blockchain state.

    \item \textbf{RPC API} - The Remote Procedure Call (RPC) interface of a Bitcoin node can be used by other clients to send commands to it that control the behaviour of the node. \cite{btcRPC}
    This interface is more powerful than the REST API since apart from allowing users to read data from the chain, it also allows them to mutate the chain's state. For eg. by making the \textit{generateblock} call which creates a new block.
    Both requests and responses from this interface are in the JSON format.
    The interface can be password protected so only authorised users can send the commands.
\end{itemize}

\newpage

% Figure \ref{fig-btc-arch} depicts the interfaces used.

\hspace{0pt}
\vfill
\begin{figure}[!htb]
    \centering
    \includegraphics[width=\textwidth]{"Protocol Interfaces".pdf}
    \caption[Protocol interfaces exposed by Bitcoin Core.]
    {
        Protocol interfaces exposed by Bitcoin Core. \\
        \footnotesize
        \textit{bitcoin-cli} can only use the RPC API to send commands, while a Python script can use either of RPC or REST.
    }
    \label{fig-btc-arch}
\end{figure}
\vfill
\hspace{0pt}


\newpage
\section{Problems with PoW Blockchains} \label{intro-problems}

As explained before, in a Proof-of-Work blockchain, every miner tries to solve a computational puzzle and the first one to do so gets to broadcast their block to the network, which the other miners accept and the block is added to the chain. 
Each block adds some weight to the chain depending on the difficulty of the puzzle that they solved. 
This strengthens the chain against future attacks.
In cases where more than one miner solves the puzzle at the same time, a fork is created. 
When other miners observe these competing chains they extend the one with the most cumulative weight (also called the Heaviest Chain Rule). 

Previous work \cite{bitcoinOriginal, Rosenfeld} has modeled the PoW consensus scheme as a Poisson process, which means that blocks get added to the blockchain at random time instants and the weight of the chain increase unsteadily. 
This causes various security and performance problems.

\textbf{Forks} - A fork is created when more than one miner create blocks pointing to the same parent. 
This can happen because they both ended up solving the puzzle simulataneously or perhaps one of the node was lagging behind the network and was not aware of the latest blocks, so ended up mining on an older block. 
These "natural" forks are distinct from forks that might be intentionally created by a malicious attacker.
A fork is resolved when one of the branches of the chain gets longer (acquires more weight) than the others.
% The shorter chain (and all blocks in it) is  
As the majority of the network switches to heavier chain, the blocks in the shorter chain become \textit{stale} and are essentially discarded. 
Miners who created the stale blocks lose their mining fees, which means their work was "wasted".
During a fork there is uncertainty in the ordering of the blocks in the chain as the network hasn't agreed on a single chain.
% This delay in consensus 
When forks take a long time to get resolved, the blockchain is more prone to attacks such as double-spending and selfish-mining \cite{selfishmining, selfishMiningStrategies}. 

A successful \textbf{double-spending} attack allows an attacker to spend the same bitcoin twice.
% To execute this, the attacker spends some bitcoin that they have on the chain, and then create a fork in the chain
To execute this, the attacker creates two conflicting transactions that both spend the same bitcoin and broadcasts it onto the network. 
To guard against these attacks, users are suggested to wait for multiple blocks to arrive on the chain before accepting a payment or releasing the goods. This duration is known as the \textit{confirmation time}.

\textbf{Selfish-mining}, on the other hand, is a block-withholding strategy where an attacker secretly mines a chain of blocks and does not broadcast it to the network. By suppressing this information the attacker tricks honest nodes into “wasting” their computational power by mining on blocks that already have successors (which are known only to the attacker). 
Once the attacker's chain gets longer than the honest chain, blocks generated by the attacker become part of the eventually agreed-upon heaviest chain, while blocks generated by honest nodes are discarded.
This attack is possible because an attacker with less mining power than the honest miners can occasionally generate a longer private chain.

One way to prevent the occurrence of natural forks is by increasing the block inter-arrival time as that gives the block a longer duration to reach all nodes in the network. 
In Bitcoin, the average time between generation of two blocks is kept to be 10 minutes to prevent natural forks from occurring;
but this also leads to \textbf{reduced throughput} and \textbf{increased confirmation times}.

Throughput of a Blockchain may be measured as the number of transactions that get on to the chain per unit time (transactions per second). 
This depends on two main factors, the block size and the block inter-arrival time. 
Bitcoin has an average block size of around 1 MB while the inter-arrival time is 10 minutes which translates to around 7 to 10 transactions per second.
This is orders of magnitudes less compared to traditional payment networks. 
For example, PayPal is capable of handling a few hundred transactions per second 
whereas VISA can process up to several thousand transactions per second. \cite{scalingBlockchains, visanet}

To prevent double-spending attacks, Bitcoin users are suggested to wait till their transaction is atleast 6 blocks deep in the chain before releasing any goods in lieu of the transaction.
This rule of thumb ensures that the chance of an attacker being able to overtake the main chain is less than a security 0.001\%. This analysis assumes that the attacker has less than 10\% of total hashing power \cite{bitcoinOriginal, Rosenfeld}.
The confirmation time using this rule comes out to one hour, which is impractical for typical scenarios that require instant payment like buying a cup of coffee.

%----------------------------------------------------------------------------------------

\chapter{Anchors} \label{ch-anchors}

In this chapter, we describe the the key contribution of this thesis - Anchors. We begin with their structure and then discuss their lifecycle - how they're generated, propagated and processed. 
We also list down some properties that Anchors have, which make them well suited to solve the problems listed in the last section (\ref{intro-problems}).

%----------------------------------------------------------------------------------------

\section{Structure} \label{anc-structure}

An \textbf{Anchor} is a small bit string consisting of 

\begin{enumerate}[i]
    \item Hash of solution block in the main chain, called its \textit{parent block}, and 
    \item a solution of a Proof-of-Work puzzle that is different (and easier) than that of a block.
\end{enumerate}

Optionally, it may contain information to reward the miner for performing the work to generate an anchor.
This can be done the same way as Bitcoin - by including a Coinbase transaction that creates fresh Bitcoins (in that it has no inputs) to reward the miner.

\paragraph{Note: } Do note that Anchors contain no other information, specifically, no other transactions. 
So they have a considerably smaller body when compared to Bitcoin blocks.

\info{Draw diagrams: block / anchor structure}

In our implementation (see Chapter \ref{ch-impl}) an Anchor is identical to an empty Bitcoin block - one that contains no transactions apart from the Coinbase.
Since anchors contain PoW, they (like blocks) also contribute weight to the chain they belong to.
%----------------------------------------------------------------------------------------

% If the content turns out to be smaller than the paper
% then we don't even have to create a separate section?

\section{Lifecycle}

\subsection{Generation} \label{anc-gen}

\begin{figure}[!htb]
    \centering
	\includegraphics[width=0.8\textwidth]{"Difficulty Targets".pdf}
    \caption[Block vs Anchor difficulty targets]{
        Block vs Anchor difficulty targets.\\
        \footnotesize
        2\textsuperscript{256} values are possible when using the SHA-256 hash function. \\
        A part of which contains valid Block hashes, while a larger \& non-overlapping part contains valid Anchor hashes.
    }
    \label{fig-targets}
\end{figure}

An Anchor is generated akin to a block but on a larger \& non-overlapping as shown in Figure \ref{fig-targets}.
This is similar to the 2-for-1 PoW technique used in \cite{backboneprotocol, fruitchain, prism}.
A miner, when mining for a block checks for each nonce value in the header, if he has found a valid block.
In case this isn't, the same solution can then be checked to see if it is valid anchor, in which case the miner publishes the anchor (along with the coinbase and a merkle path of the coinbase to the root of the tree.) 
{\em This method ensures that no additional effort is needed for mining an anchor}. 
The miner can simply continue the mining process in search of the next block (or anchor) on the chain. 

% The Honest miner broadcasts an Anchor to its peers as soon as it is generated and updates the weight of his chain locally.

The Merkle path that is additionally sent, is needed to prove to a receiver that the coinbase contained in the Anchor actually belongs to the Merkle tree of transactions whose root is stored in the header \cite{bitcoinOriginal}. 
The value in the 'hash of the previous block' field of the published anchor is the hash-pointer to the parent block of the anchor. 
It is important to note that, the other values in the header like difficulty targets and the coinbase were originally intended for the next block and since a trial solution became an anchor these fields could not be updated. 

\subsection{Propagation} \label{anc-prop}

Once an Anchor has been found, it needs to be propagated throughout the network.
Published Anchors propagate through the P2P network similar to transactions and blocks. 
The small size of Anchors (as compared to blocks) allows them to propagate faster throughout the network. 
The effects of including Anchors with respect to propagation times are discussed in section \ref{exp-prop} while the results from the experiments are in Sections \ref{res-prop-1}, \ref{res-prop-2}. 

\subsection{Processing} \label{anc-proc}

A node upon receiving a valid Anchor updates the weight of the chain(s)
\footnote{More than one chain is possible when the received Anchor points to an older Block and since then the chain has been forked} it belongs to. 
The node then forwards the Anchor to all its neighboring peers. 
If the revised weight introduced by the new Anchor causes a switch in case of a fork, as calculated by the Heaviest Chain Rule, the node shifts to the new chain tip and continues mining on that chain.

%----------------------------------------------------------------------------------------

\section{Properties} \label{anc-properties}

We now list some properties of anchors that make them suitable to solve the problems listed in section \ref{intro-problems}.

\begin{itemize}
    \item \textbf{Higher Frequency: } 
        Anchors are generated using a larger PoW target size than blocks. 
        They are hence generated at a faster rate adding weight more smoothly to the chain than blocks. 

    \item \textbf{Faster Propagation: }
        Anchors are fixed size, small structures containing minimal data (only the header and optionally coinbase with no other transactions). Hence, Anchors are light weight and propagate faster in the network than blocks.
    
    \item \textbf{Faster Verifcation: }
        Since Anchors do not contain transactions, they can be easily verified in O(1) time. 
        A node only needs to check whether a received Anchor satisfies the PoW difficulty target. 
        While, for a block, all transactions contained within it must also go through a verifcation process - checked for double-spends etc.
    
    \item \textbf{Non-Forking: }
        Anchors cannot create new forks in the chain, unlike blocks, as no block or anchor can point to another Anchor. Consequently, every successfully mined and published Anchor may contribute to the weight of the branch(es) its parent block is in.
    
    % \item \textbf{Non Precomputable: }
    %     Anchors contain a pointer to a recent block on the main chain on which they were mined on. Hence they are not precomputable.

    \item \textbf{No extra hashing power: } 
        Since Blocks and Anchors are simultaneously mined on non-overlapping target spaces as shown in Figure \ref{fig-targets}, Anchors do not require extra hashing power to generate.

\end{itemize}

%----------------------------------------------------------------------------------------

\section{Benefits} \label{anc-benefits}

The above mentioned properties make Anchors a useful tool in solving various problems and lead to the following benefits:

\begin{itemize}
	\item \textbf{Faster Fork Resolution:} 
        Anchors provide a fast signaling mechanism for miners to know which chain has a majority of the hashing power of the network. 
        In case of a fork, the system has to wait only for the first Anchor to resolve it, rather than the next block like Bitcoin.
        This matters because Anchors are generated at a higher rate than Blocks so arrive more frequently.
        % This gives them insight into the surviving chain of a fork  as shown in Figure \ref{fig:forkresolution}. 
        %Anchors provide fast signaling of hash power working on the chain. 
        %Miners can use this information to decide on which chain to extend thus resolving forks faster. 
        % This feature is especially useful in systems with high fork rates~\cite{GHOST,EthereumOriginal,spectre,conflux,phantom}
	
        %Miners can simply switch to a heavier branch determined by the weight contributed by anchors thereby resolving forks much faster than traditional blockchains without anchors as shown in Figure \ref{fig:forkresolution}. 
        %Anchors provide the means for miners to know where the majority of hashing power of the network lies, which also helps in quick resolution of forks.
        % Anchors help attain stability much faster when the system is threatened especially when the block interval time is particularly large to resolve a fork and an attacker has a long window to take advantage of the division of honest power.
	
    \item \textbf{Increased Stability and Reduced confirmation time:} 
        Anchors increase stability with a steady contribution to the weight of the chain which makes selfish mining attacks harder. 
        Consequently they reduce the confirmation time of transactions and make double-spends more difficult.
        %  as shown in Figure \ref{fig:}
		
        %	\item \textbf{Reduces chance of selfish mining and confirmation time:} The steady growth on of weight on the honest chain significantly reduce chances of selfish mining attacks with lower confirmation time guarantees.
	
        %	\item \textbf{Avoids Double Spends:} Double spend attacks can be avoided at a much earlier stage with the help of Anchors as chances of selfish mining are greatly reduced.

    \item \textbf{Eliminates need to join Mining Pools:} 
        Over time, Bitcoin mining has become centralized in that a few Mining Pools control a large portion of the mining power.
        Miners are incentivised to join a pool since Block solutions are scarce and being able to find one is hard for a lone miner.
        As Anchors are generated at a faster rate than blocks, these small miners can reap the anchor rewards more smoothly over time, thus reducing the need for them to join the pools. 
        
    %\vinay{What you have said below is very important. Only problem is that we have not yet quantified how anchors help scaling up throughput. Also, there are tons of other methods for scaling up throughput. We should probably just stick to discussing stability.}
    % \item \textbf{Contributes to Scalability of Bitcoin:} 
    %     The problem of increasing Bitcoin’s transaction throughput is known as the Scalability problem. 
    %     The two main options to tackle this are (i) to increase the size of blocks, or (ii) to decrease the block intervals. 
    %     Both options lead to an undesirable outcome: an increase in the rate of forks. 
    %     The block-size/block-interval parameter adjustment is a difficult line to toe, as is clear from the tenor of the scalability debate. With the help of anchors helping solve the forking problem, we can even look at these previous approaches to scalability from a new perspective~\cite{bitcoinNG,EthereumOriginal}
	
\end{itemize}

%----------------------------------------------------------------------------------------


\chapter{Implementation} \label{ch-impl}

\section{Building Bitcoin Core} \label{impl-btc-compile}

As mentioned before, the reference implementation of Bitcoin (known as Bitcoin Core) is written in cross-platform C++ and is available on all major operating systems like Windows, macOS and various flavours of Linux \cite{bitcoinDownload}. Since we had to make changes to the Core code we used the freely available source code of version 0.16 \cite{bitcoinSource}. The code uses GNU Build System \cite{gnuBuild} to automate the build process. We now describe the process of compiling Bitcoin Core. 

\subsection*{Installing Dependencies}

The first step is to install all required dependencies, which varies from platform to platform. Listings \ref{code-btc-deps-ubuntu}, \ref{code-btc-deps-arch} show how to install them on Ubuntu 16.04 and Manjaro 18.0 respectively.

\begin{listing}[!htb]
    \begin{minted}[]{bash}
#!/bin/bash

$ sudo add-apt-repository ppa:bitcoin/bitcoin
$ sudo apt-get update

$ sudo apt-get install build-essential libtool autotools-dev \
    automake pkg-config libssl-dev libevent-dev bsdmainutils \
    python3 libboost-system-dev libboost-filesystem-dev \
    libboost-chrono-dev libboost-program-options-dev \
    libboost-test-dev libboost-thread-dev libdb4.8-dev \
    libdb4.8++-dev protobuf-compiler libprotobuf-dev

$ sudo apt-get install libqt5gui5 libqt5core5a libqt5dbus5 \
    qttools5-dev qttools5-dev-tools libprotobuf-dev \
    protobuf-compiler libqrencode-dev

    \end{minted}

    \caption[Installing dependencies required for Bitcoin Core on Ubuntu 16.04]
    {
    Installing dependencies required for Bitcoin Core on Ubuntu 16.04 \\
    \footnotesize
    Commands taken from the original documentation \cite{bitcoinBuildUnix}.
    }
    \label{code-btc-deps-ubuntu}
\end{listing}


\begin{listing}[!htb]
    \begin{minted}[]{bash}
#!/bin/bash

$ sudo pacman -S boost libevent qt5-base qt5-tools qrencode \
    miniupnpc protobuf zeromq db4.8

\end{minted}

\caption[Installing dependencies required for Bitcoin Core on Manjaro 18.0]
{
    Installing dependencies required for Bitcoin Core on Manjaro 18.0 \\
    \footnotesize
    Commands adapted from Bitcoin PKGBUILD file for ArchLinux \cite{bitcoinBuildArch}.
    }
    \label{code-btc-deps-arch}
\end{listing}

\subsection*{Compiling the source}

Once all required dependencies have been installed, we download the source code from the official repository on GitHub and proceed to compile it. Listing \ref{code-btc-compile} lists the steps. 

\begin{listing}[!htb]
    \begin{minted}[]{bash}
#!/bin/bash

# Get the source from official repository on GitHub
$ git clone https://github.com/bitcoin/bitcoin
$ cd bitcoin

# We used v0.16 of the source
$ git checkout v0.16

# Disable parts of the code that we don't require to speed up compilation
$ ./autogen.sh
$ ./configure --disable-tests --disable-bench \
    --without-gui --with-incompatible-bdb

# Compile using 4 CPU cores
$ make -j4

    \end{minted}

    \caption[Compiling Bitcoin Core]
    {
    Compiling Bitcoin Core \\
    \footnotesize
    Commands taken from the original documentation \cite{bitcoinBuildUnix}.
    }
    \label{code-btc-compile}
\end{listing}

To speed up the compilation process we pass the following flags to \textit{GNU Configure}:

\begin{itemize}
    \item \textbf{--disable-tests} - Skip compiling tests written Bitcoin Core.
    \item \textbf{--disable-bench} - Skip compiling benchmarks.
    \item \textbf{--without-gui} - Skip compiling the Qt based GUI.
    \item \textbf{--with-incompatible-bdb} - Allow using a bdb version other than 4.8
\end{itemize}

%----------------------------------------------------------------------------------------

\newpage
\section{Anchors with Bitcoin} \label{impl-anchors}

We decided to implement Anchors in Bitcoin as it is the most widely used Proof-of-Work Blockchain. We used the stable version (v0.16 released on 26-02-2018) of the reference Bitcoin implementation (also called "Bitcoin Core") that was available to us before we started working on this project. For reference, the latest stable release of Bitcoin is v0.18 and was released on 18-05-2019. 

As described in section \fixme{Add reference to correct section here.} \textit{Bitcoin Core} exposes a Remote Procedure Call (RPC) based API that other clients can use to control a Bitcoin node's behaviour. We implement Anchor support in Bitcoin by adding two new RPCs: \textbf{submitanchor} \& \textbf{generateanchor}. We now explain the working of these RPCs.

%----------------------------------------------------------------------------------------

\subsection{submitanchor} \label{impl-submitanchor}

The \textit{submitanchor} RPC is modelled around the already existing \textit{submitblock} RPC. 
It takes in as an argument a serialized representation of an anchor (which is generated outside of Bitcoin Core), ensures that it is valid, updates its local Blockchain, and then broadcasts the anchor to its peers where the process continues.

\subsubsection{Validating an Anchor}

Since we planned to test our implementation in a controlled environment; where all clients only send valid anchors, we have not implemented any validation checks for anchors. 
We only check whether an Anchor has already been seen before, in which case, we ignore such duplicate Anchors.
In a real-world implementation, further checks like ensuring that the Block pointed to by an anchor has already been received etc. would have to be performed before an anchor is considered valid \& broadcasted.

\subsubsection{Updating the local Blockchain}

When an anchor has been deemed valid, its effects need to be updated onto the local copy of the Blockchain.

Bitcoin Core represents the blockchain in memory as a 'parent pointer tree' - which is an N-ary tree where each node contains a pointer to its parent node, but no pointers are stored to child nodes. 
This structure has worked for Bitcoin's blocks as each Block either extends an existing chain or creates a fresh one (by forking it) so references to child nodes are never required.
However, an anchor arriving on a block modifies the weights of all descendants of the block which requires us to also store children information in each block to be able to efficiently walk the "Block-tree".

Figure \ref{fig-impl-parent-pointer} depicts Bitcoin's original design of the Blockchain, whereas 
Figure \ref{fig-impl-child-pointer} shows the updated design of the chain where blocks also store a list of children.

\begin{figure}[!htb]
    \centering
    \includegraphics[width=\textwidth]{"Parent Pointer".pdf}
    \caption{Bitcoin's original blockchain representation}
    
    \medskip
    \footnotesize
    Single pointed arrows depict that a block only stores a pointer to its parent.
    % As explained in Section 
    \label{fig-impl-parent-pointer}
\end{figure}

\begin{figure}[!htb]
    \centering
    \includegraphics[width=\textwidth]{"Child Pointer".pdf}
    \caption{Modified blockchain representation}
    
    \medskip
    \footnotesize
    Double pointed arrows depict that a block stores pointers to its children as well.
    % As explained in Section 
    \label{fig-impl-child-pointer}
\end{figure}

Listing \ref{impl-cblockindex} shows the change made to the core data structure that stores the blocks in memory. Specifically, a new list of child pointers is added and is maintained along with the parent pointers.

\textbf{Note:} Do note that these changes have been made to the in-memory representation of a Block; they're only stored locally, and are not propagated across the network. Each Bitcoin node maintains this information locally.

\begin{listing}[!htb]
    \begin{minted}[]{cpp}
        class CBlockIndex // {
            public:

            // pointer to the hash of this block
            const uint256* phashBlock;

            // pointer to the index of the parent of this block
            CBlockIndex* pprev;

            // pointers to the children of this block
            std::vector<CBlockIndex*> children;

            // height of this block in chain
            // (the genesis block has height 0)
            int nHeight;

            // total amount of work (expected number of hashes) 
            // in the chain up to and including this block
            arith_uint256 nChainWork;

            ...
            
        }
    \end{minted}

    \caption[Updated CBlockIndex which also stores a list of child pointers.]
    {
        Updated CBlockIndex which also stores a list of child pointers.
        % \medskip
        \footnotesize
        Line 11 has been added. \\
        Snippet of code taken from Bitcoin Core source file \textit{src/chain.h}.
    }
    \label{impl-cblockindex}
\end{listing}


% In the current implementation, Anchors are not stored to disk
% because all our tests were performed on a fresh Blockchain 

We have not implemented storage / retrieval of anchors to disk because our tests created a fresh chain every time they ran. 
In a real-world implementation this would also have to be added so that changes can persist after events like client shutdown, machine restart etc.

\newpage
\paragraph{Updating weight of the chain:} 

\info{Each node computes the weight with information it has locally; nChainWork}

An anchor updates the weight of the block that it points to and of all its descendants.
This is done by performing a Breadth-first-search originating from the parent block and continuing down the tree.

Figure \ref{fig-impl-anchor-weight-update} depicts the scenario of an anchor arriving on a Block and the blocks whose weights would have to be updated.

% Hack to hide overfull hbox warning
\setlength{\emergencystretch}{10pt}

\begin{figure}[!htb]
    \centering
    \includegraphics[width=\textwidth]{"Anchor Weight Update".pdf}
    \caption[Weight update due to an Anchor]
    {Anchor arriving on a Block updates its weights and of all its descendants (shown highlighted.)}
    
    \medskip
    \footnotesize
    The double pointed arrows depict that a block also stores pointers to its children. 
    % As explained in Section 
    \label{fig-impl-anchor-weight-update}
\end{figure}

\paragraph{Anchors-per-Block:}
We added a new parameter called \textit{anchorsperblock} that controls the amount of weight that an anchor adds to the chain. The parameter is set once when the Bitcoin process starts. 
As an example, An \textit{anchorsperblock} value of 10 specifies that 
anchor generation is ten times easier than block generation, so ten anchors are generated for every block and in such a case, an anchor adds 1/10th of the weight that a Block adds to the chain.

%----------------------------------------------------------------------------------------

\subsubsection{Brodcasting an Anchor to the network}

The RPC call to submit an anchor happens on the TCP interface of Bitcoin Core which uses JSON as its data representation format. As explained in Section \fixme{ref} the Bitcoin Protocol uses a separate interface to exchange messages between the network. 
We add a new message type that uses the same Bitcoin Protocol to send Anchor messages. 
See the source file \textit{src/protocol.cpp} \& \textit{src/protocol.h} for the relevant code.

Anchors are propagated in the network by flooding, wherein each node broadcasts Anchor message to all its immediate neighbors (which process the anchor and further brodcasts it.)

% TODO: Add information about separate threads and passing data among them.

\paragraph{Dealing with multiple threads:} 
The two interfaces that Bitcoin Core provides (JSON-RPC \& BitP) operate on separate network sockets and are handled by separate threads of the process. 
Data is passed among these threads via a signal \& slot architecture, where each thread has handlers that listen on pre-defined slots, waiting for an event (a signal) to occur. When the signal fires, the slot handling function is executed.
We follow the same design and create a new signal \& the corresponding slot for Anchors called \textbf{AnchorConnected} which performs the actual broadcast of Anchor messages.
See the source file \textit{src/net\_processing.cpp} \& \textit{src/validationinterface.cpp} for the relevant code.

%----------------------------------------------------------------------------------------

\subsection{generateanchor} \label{impl-generateanchor}

The \textit{generateanchor} RPC is modelled around the already existing \textit{generate} RPC (which is used to generate a block). 
While the \textit{submitanchor} RPC receives an anchor's representation as an argument, the \textit{generateanchor} RPC takes no arguments. When it is called, a new anchor is generated by the Bitcoin node itself, and is processed in the same way as explained in section \ref{impl-submitanchor}.

We added this RPC because \fixme{continue this}

% 3.2 Test Bed for Experiments
% * Master / Slave architecture
%     * 3 generations?

% * Txn Generation
%     * Master used to send to all
%     * Separate script
%         * sendtoaddress
%         * Tried using jmeter
%         * sendmany

\newpage
\section{Testbed} \label{impl-testbed}

In this section we describe the implementation of a testbed that can be used to perform experiments on the Bitcoin blockchain. Some salient features of our testbed are:

\begin{itemize}
    \item Spawn a network of Bitcoin nodes running on a cluster of cloud-based virtual machines.
    \item Replace Proof-of-Work with a centralized control program that triggers Blockchain events like generation of blocks \& anchors.
    \item Emulate conditions that a real-world blockchain faces by genenrating a high throughput of transactions, introducing network latencies etc.
    \item Fine grained control over various parameters controlling the experiment.
    \item Process the logs generated by the Bitcoin nodes.
\end{itemize}

The Bitcoin Core code ships with a library (called \textit{TestFramework}) that can spawn a network of Bitcoin nodes; but all those nodes can only run on a single machine. We saw this as a limitation as it prevented us from scaling up our experiments to hundreds of nodes. 
To remedy this, we extended the library by adding the ability to run Bitcoin nodes on a remote virtual machine (using the SSH protocol.)

Though Bitcoin Core is written in C++, the \textit{TestFramework} library is written in Python. To be able to use the same codebase, we implemented our tested in Python (v3.7) as well. 
This allowed us to utilize the vast ecosystem of third-party libraries that Python has, some of which are: 

\begin{itemize}
    \item \textbf{paramiko} for connecting to remote machines using SSH.
    \item \textbf{socket, http.client} for estabilishing RPC connections to Bitcoin nodes.
    \item \textbf{json} for serializing the data being sent over the RPC connection.
    \item \textbf{random} for generating random events (like block or anchor generation etc.) that resembled an exponential probability distribution.
    \item \textbf{time} for "sleeping" between these events, instead of performing proof-of-work computation. 
\end{itemize}

%----------------------------------------------------------------------------------------

\newpage
\subsection{Centralized Controller} \label{impl-master}

The core part of the testbed is a centralized program (called "Master") that controls the entire experiment. We now explain the various tasks that the Master program performs.

% \vspace{0.5cm}

\subsubsection{Network Setup}

\begin{enumerate}
    \item The Master program begins by launching the Bitcoin processes on remote virtual machines (VMs) using SSH.
          This can be configured by specifying the IP addresses of the VMs and the number of Bitcoin nodes to launch on each VM.
          Each Bitcoin process locks up some system resources like the port numbers it listens on for RPC / Bitcoin protocol connections, and the "data-directory" it uses to store its local blockchain \& log files.
          To launch multiple Bitcoin processes on a single machine, we allocate a different set of such resources for each process by passing in arguments while launching the process.

    \item Once all the processes have been launched, a TCP/RPC connection is acquired to each one of them. 
          This connection is used to send RPC messages to the process to control its behaviour.

    \item By default, a Bitcoin process has no idea of the other ones running in the network, so the Master program then connects these processes into a network according to a predefined topology by sending \textit{addnode} RPC command.
          Further discussion of the network topology is deferred to Section \fixme{ref}.
\end{enumerate}
      
\subsubsection{Chain Setup \& Transaction Generation}

% Though real-world blockchains start their operation from a

\begin{enumerate}
    \item Once the nodes become aware of each other, the Master program generates empty blocks at each node so they have bitcoins to spend.
    
    \item These bitcoins are later used to generate transactions that are used to fill up the Blocks that are generated.
          Secction \ref{impl-transactions} goes over this process in more detail.
\end{enumerate}

\subsubsection{Block \& Anchor Generation}

\begin{enumerate}
    \item While generating a Block or an Anchor the Master program avoids doing actual proof-of-work computation and instead simulates the mining process by randomly choosing a node as the creator of a Block (or Anchor) at random time instants.
    Since Block inter-arrival times in Bitcoin can be approximated by an exponential distribution \cite{bitcoinOriginal}, the random time instants are sampled from such a distribution with a predefined mean.
    The inter-arrival times between Anchors are sampled from another exponential distribution with a different mean.

    \item This process continues for a a predefined amount of time (called the "lifetime" of the experiment) after which Block \& Anchor generation is stopped.
\end{enumerate}

\subsubsection{Log Processing}

Each Bitcoin process logs all the events that happen on the node into a log file stored in the "data-directory".
% We added new log lines
Once the experiment finishes, the Master program fetches logs from all the nodes, extracts relevant information that is of use to us and generates a result file for the experiment which contains the various statistics that we measure. Refer to the \fixme{ref} for the results that are generated.

\subsubsection{Cleanup}

After the logs have been processed, the Master program stops all the running Bitcoin processes, deletes the data-directories and removes all manually introduced delays from the nodes so that the next experiment can start from a clean slate.

\info{Add snippets of code of threads that generate Blocks, Anchors, Transactions?}

%----------------------------------------------------------------------------------------

\newpage
\subsection{Architecture} \label{impl-testbed-arch}

We went over different architectures for our testbed, starting with the most straightforward - where the Master program handled the bulk of the load during the experiments, and later moving on to a setup where the load was distributed amongst multiple processes running on different machines. We now go over details of various iterations of our architecture, explain what problems we faced and what we did to resolve them.

\subsubsection{Iteration 1} \label{impl-arch-1}

We began with a setup where the Master was run on one virtual machine (VM) and it connected to Bitcoin processes running on other VMs via a TCP-RPC connection.

\begin{figure}[!htb]
    \centering
    \includegraphics[width=\textwidth]{"Arch v1".pdf}
    \caption{First iteration of testbed architecture}
    
    \medskip
    \footnotesize
    The "Master" program controls generation of Blocks, Anchors and Transactions.
    \label{fig-impl-arch-1}
\end{figure}

Figure \ref{fig-impl-arch-1} depicts the architecture: On the left is the Master program (running on a single VM) and on the right is the rest of the Bitcoin network running on a cluster of VMs, each of which runs multiple Bitcoin nodes which have been connected to each other according to a predefined topology.

The Master program consists of three separate threads of execution that deal with generation of Blocks, Anchors \& Transactions. 
These threads work by randomly picking a Bitcoin node from the network and sending an appropriate RPC command to it, and then waiting for some time before sending the next command.

The main problem we faced with this architecture was trying to generate a high throughput of transactions (which was required to have fully filled Blocks.) We observed that as we tried to increase the throughput, a majority of the RPC calls would fail due to socket timeouts, leading us to believe that the Master was creating a bottleneck in the system.

%----------------------------------------------------------------------------------------

\subsubsection{Iteration 2} \label{impl-arch-2}

We decided to extract out the transaction generation thread from the Master program, convert it into a separate script and move it to each of the individual VMs. 

\begin{figure}[!htb]
    \centering
    \includegraphics[width=\textwidth]{"Arch v2".pdf}
    \caption{Second iteration of testbed architecture}
    
    \medskip
    \footnotesize
    Transaction generation thread has been moved to each VM to increase throughput.
    \label{fig-impl-arch-2}
\end{figure}

Figure \ref{fig-impl-arch-2} shows the updated architecture, where a "Generate Transactions" script is run on each VM. The script only connects to Bitcoin processes running on its own machine so that all socket connections act as local (or loopback) connections. This reduces the network overhead of sending RPC calls from the Master program. 

This architecture fixed the problem with transaction throughput, and we continued to use it until we needed to scale our setup to a large number of nodes (100+) with the experiments having shorter inter-arrival times between the various events (Blocks, Anchors, Transactions). When we tried to generate an anchor every second, we observed that some of the RPC calls would fail due to socket errors.

%----------------------------------------------------------------------------------------

\subsubsection{Iteration 3} \label{impl-arch-3}

Similar to what we did before, for our final iteration of the architecture, we moved the Block and Anchor generation threads from Master to the individual VMs.

\begin{figure}[!htb]
    \centering
    \includegraphics[width=\textwidth]{"Arch v3".pdf}
    \caption{Third iteration of testbed architecture}
    
    \medskip
    \footnotesize
    Block \& Anchor generation threads have also been moved to each VM.
    \label{fig-impl-arch-3}
\end{figure}

Figure \ref{fig-impl-arch-3} shows this architecture. Apart from a transaction generating script, a block and anchor genenrating script has also been added to each VM \fixme{continue}.

The master program is now only responsible to setup the network, and launch its various components. After which, it sits idle waiting for the experiment to end. We've used this architecture for our final set of experiments.

%----------------------------------------------------------------------------------------

% \newpage
\subsection{Transaction Throughput} \label{impl-transactions}

In the real-world, Bitcoin blocks contain transactions which are generated by millions of people throughout the world when they try to make payments for various goods and services. \info{Txn structure?} To emulate this process in our experiments, we generated dummy transactions between various Bitcoin nodes in the network, where one node randomly picks another and sends it a few bitcoins.

% Could be added after 1st sentence?
% \begin{itemize}
%     \item Image showing the structure of a Bitcoin transaction, and its description?
%     \item Components: Address, Amount, Input, Output.
%     \item Snippet of code of the sendtoaddress RPC.
% \end{itemize}

To keep our testbed as close to the real-world as possible, we wanted to generate enough transactions per second to fill up all Blocks that are generated. A typical Bitcoin block is of 1 MB in size whereas the smallest transaction, which transfer bitcoins from one node to another (and hence contains one input and one output) is roughly of 300 bytes, so around 3500 transactions are needed for every Block. This is easier when the inter-arrival time between blocks is larger. In Bitcoin, a new block is generated (roughly) every ten minutes, which is enough for transactions to be generated to fill the Blocks. However, we wanted to test scenarios where the block-interval time is short (say in the order of ten seconds) so we had to try various approaches to increase the transaction throughput of the system.

\subsubsection{sendtoaddress RPC}

As mentioned in Section \ref{impl-arch-1} we started with an architecture where the Master program genereated all the transactions by sending \textit{sendtoaddress} RPC to one of the nodes at random time intervals. Listing \ref{code-testbed-sendtoaddress} shows the Master's transaction generating thread. 
We sample two nodes at random, arbitrarily assign one of them to be the payer and another to be the payee.
The \textit{sendtoaddress} RPC is sent to the payer, which generates the transaction and broadcasts it throughout the network.

\begin{listing}[!htb]
    \begin{minted}[]{python}

while True:
    
    # Select two random nodes 
    payer, payee = random.sample(NODES, 2)

    payees_address = address_of[payee]
    amount_to_pay = 0.01

    # Send the RPC call to the node that will pay
    payer.sendtoaddress(payees_address, amount_to_pay)

    \end{minted}

    \caption[Generating transactions using the \textit{sendtoaddress} RPC]
    {
        Generating transactions using the \textit{sendtoaddress} RPC.

        \footnotesize
        One node sends a token amount of bitcoins to another.
    }
    \label{code-testbed-sendtoaddress}
\end{listing}

The \textit{sendtoaddress} RPC creates the simplest form of Bitcoin transaction, having exactly one input and exactly one output. As mentioned above, we need 3500 of such transactions per Block, which for a Block inter-arrival time of 30 seconds comes out to around 120 transactions per second.

In practice we were only able to generate around 10 transactions per second, which is only one-tenth of what was required. Even after we moved to a setup where the transaction generation process was remove from Master and distributed to the remote virtual machines, we did not observe any improvement in transaction throughput.

\subsubsection{Apache JMeter}

At first, we attributed this low throughput of transactions to our implementation of the Master program, in particular the transaction generating thread, which was in Python and used Python's in-built socket \& http libraries. We felt that using a compiled language might give us better results, but since porting the code to another language would take up a lot of time, we instead decided to use an already existing application that would be well suited for this task - Apache JMeter \fixme{cite Apache JMeter homepage} - which is a Java based open-source application that is used to load-test and measure performance of other applications. It supports a wide variety of web based protocols like HTTP, FTP, Generic TCP, LDAP etc.

We used JMeter in a distributed setup \fixme{cite Apache docs page?} since it was appropriate for our setup. We ran JMeter server on one of the VMs (which would act as master) and JMeter clients on other VMs (acting as slaves.) We created a JMeter test plan which uses the \textit{HTTPSamplerProxy} class to send JSON-RPC commands to the Bitcoin nodes running on the VM. JMeter allows the load on machines to be configured via multiple methods. We used a \textit{ThreadGroup} with 100 threads running on each VM with a "ramp up" time of 0 seconds, which essentially means that we expect 100 transactions to be generated per second, but in practice we observed similar results as before - only 10 transactions per second were actually being generated, which meant that the low throughput issue is not with our implementation of Master but instead lies within Bitcoin itself.

JMeter logs the time each RPC call took to finish execution and return its result. Analyzing those logs gave us insight into the real problem. We observed that the inital RPC calls executed faster taking only 0.5 seconds to generate a transaction, while as time went on the response time increased, with some of the calls taking as much as 100 seconds to generate a transaction. 

To understand the root cause of the problem, we read the source code involved in generating a single transaction. We found that the \textit{sendtoaddress} RPC is fundamentally limited in the way it chooses the bitcoins to spend to generate a transaction and as the mempool becomes larger new transactions are generated slowly. This has not been a problem for the main Bitcoin network because the transaction throughput of the system is low by design.

\info{Mention the ancestor limit; explain?}

While going through the source code of Bitcoin's RPC server implementation (located in \textit{src/httpserver.cpp}) we also discovered some parameters (passed onto the process when it is launched) that control behaviour of the RPC server:

\begin{itemize}
    \item \textbf{rpcthreads} - the number of threads to service RPC calls. By default 4 threads are run, but increased them to 8.
    
    \item \textbf{rpcworkqueue} - the depth of the work queue used to service RPC calls. The default value is 16, but we increased it to 64 because we found that further increasing it didn't improve performance.
    
    \item \textbf{rpcservertimeout} - timeout during HTTP requests. We left this to its default value of 30 seconds since we want each request to succeed during that time.
\end{itemize}

%----------------------------------------------------------------------------------------

\subsubsection{sendmany RPC}

\textit{sendtoaddress} is not the only way of generating a transaction in Bitcoin. Some other ways are:

\fixme{add ref for these?}

\paragraph{raw transaction API: } 

The "raw transaction API" was introduced in Bitcoin version 0.7. It gives users low-level access to transaction creation and broadcast. The API has a set of RPCs that can be used to create a transaction: \textit{createrawtransaction}, \textit{fundrawtransaction}, \textit{signrawtransaction}, \textit{sendrawtransaction}. Listing \ref{code-testbed-rawtxn} shows the right order of executing these calls.

\begin{listing}[!htb]
    \begin{minted}[]{python}

while True:

    # Select two random nodes 
    payer, payee = random.sample(NODES, 2)

    payees_address = address_of[payee]
    amount_to_pay = 0.01

    # Send all RPC calls to the node that will pay
    r = payer.createrawtransaction([], { payee_address: amount_to_pay })
    f = payer.fundrawtransaction(r)
    s = payer.signrawtransaction(f["hex"])
    t = payer.sendrawtransaction(t["hex"])

    \end{minted}

    \caption[Generating transactions using the "raw transaction" API]
    {
        Generating transactions using the "raw transaction" API.

        \footnotesize
        Four RPC calls are required to create one transaction.
    }
    \label{code-testbed-rawtxn}
\end{listing}

\paragraph{sendmany: } 

While the \textit{sendtoaddress} RPC can only send bitcoins to a single address, the \textit{sendmany} RPC is used to send bitcoins to multiple addresses in a single transaction. A transaction paid to a single address is of 300 Bytes, but one having multiple addresses can be made arbitrarily large by increasing the number of output addresses. Listing \ref{code-testbed-sendmany} shows our usage of the \textit{sendmany} call. We select a random set of nodes that are paid a token amount of bitcoins. 

\begin{listing}[!htb]
    \begin{minted}[]{python}

while True:

    # Select a random node as payer
    payer = random.choice(NODES)

    # Select a random set of payees having a size in the range [5, 50)
    payees = random.sample(NODES, random.randint(5, 50))

    amount_to_pay = 0.01

    # Send the RPC call to the node that will pay
    payer.sendmany("", { address_of[p]: amount_to_pay for p in payees })

    \end{minted}

    \caption[Generating transactions using the \textit{sendmany} RPC]
    {
        Generating transactions using the \textit{sendmany} RPC.

        \footnotesize
        Can be used to create transactions of arbitrary size, since they have multiple output addresses.
    }
    \label{code-testbed-sendmany}
\end{listing}

We finally used the \textit{sendmany} RPC instead of the \textit{raw-transaction} API as it only involves sending one RPC call instead of four to generate a transaction. A typical transaction generated by our implementation was of around 5 KBs which means we only require 200 transactions per Block or 7 transactions per second (assuming Block inter-arrival time of 30 seconds) which is well within reach of Bitcoin's RPC server.


%----------------------------------------------------------------------------------------

\chapter{Experimental Evaluation} \label{ch-exp}

%----------------------------------------------------------------------------------------

\section{Setup} \label{exp-setup}

\begin{itemize}
    \item Baadal Machines
    
    \item Network Topology etc.
    \begin{itemize}
        \item Mention somewhere that addnode RPC is used
        \item Ensure that it is a single connected component?
        \item Show an image that has a few nodes? (networkx)
    \end{itemize}
    
    \item Delays (along with exact table sent to Ovia)
\end{itemize}

%----------------------------------------------------------------------------------------

\newpage
\section{Fork Resolution Times} \label{exp-fork}

\begin{itemize}
    \item How was it calculated
    \item Result Graph
\end{itemize}

%----------------------------------------------------------------------------------------

\newpage
\section{Propagation Times} \label{exp-prop}

\begin{itemize}
    \item How was it calculated
    \item Result Graph
\end{itemize}

%----------------------------------------------------------------------------------------

\newpage
\section{Chainweight Growth} \label{exp-chainweight}

\begin{itemize}
    \item How was it calculated
    \item Result Graph
\end{itemize}

\fixme{Also need to mention mean/median Block Size ~ 1.1 MB}

\chapter{Results} \label{ch-results}

% Make all sections in this chapter to be smaller
\titleformat*{\section}{\small\bfseries}

We conducted all experiments in 4 configurations, controlled by the \\ \textit{anchorsperblock} parameter, so a system without anchors (a = 0), and systems with anchor frequencies per block a = 2, 5 and 10. 
Each configuration was run 15 times for one hour, with block inter-arrival time fixed to be 30 seconds. So the expected number of blocks in each run were about 120 (60 minutes / 30 seconds).

\section{Comparing propagation times of blocks in the presence of Anchors vs propagation time of Anchors}
For the anchor structure implemented, we see that all anchors have a fixed size of 264 Bytes. The median block size observed across all runs of our experiments was 
932KB. 
Comparing average propagation times of Blocks vs Anchors, we observed that anchor propagation times are at least 
5 times faster than blocks. Our hypothesis that an anchor will propagate much faster than a block is proven here. Figure \ref{fig-prop-times} plots the observed results. Hence anchors do work as a signaling scheme to give early insights to miners about the hashing power division in the network.

\begin{figure}[!htb]
    \centering
    \includegraphics[width=0.6\textwidth]{"Propagation Times".pdf}
    \caption{Mean propagation times of anchors, and blocks with and without the presence of anchors }
    \label{fig-prop-times}
\end{figure}

\section{Comparing propagation times of blocks with and without the presence of anchors}

Figure \ref{fig-prop-times} plots the propagation time of blocks with and without anchors. We observe that these two are comparable, with propagation times with anchors increasing only 
0, 
5 and 
17 percent for $a$ = 2, 5 and 10. We can thus say that anchors may work with the system without creating bandwidth or latency overheads. 

\section{Frequency of Anchors per block vs Resolution Time of Forks (RTF)}

We observe that anchors reduce the mean fork resolution time by 70, 82 and 96 percent, with a = 2, 5 and 10 respectively. 
Figure \ref{fig-forks-rtf} shows the results of average resolution time of forks we saw in our network while varying anchor frequency (a) to 2, 5, and 10 when compared to the Bitcoin network without anchors. 

\begin{figure}[!htb]
    \centering
    \includegraphics[width=0.6\textwidth]{"Fork Resolution Times".pdf}
    \caption{Mean Resolution Time of Forks (RTF) for different values of a along with 95\% confidence intervals}
    \label{fig-forks-rtf}
\end{figure}

\section{Unition of mining power in case of forks with anchors}

Forks are resolved at each miner, based on the chain he observes locally. 
We found that, for every fork, the branch chosen by the miner to resolve it was the same across the network. Thus anchors succeed in reuniting mining power (divided across the branches of a fork) to the same chain.

\section{Frequency of Anchors per block vs Number of forks prevented}

For this experiment, we calculate the number of forks prevented as the number of miners on which the fork is resolved before occurring due to anchors. 
Figure \ref{fig-forks-prevented} shows the average number of forks prevented when varying $a$ between 2, 5 and 10, and comparing it to a system without anchors. We observe that the inclusion of anchors prevents forks from occurring.

\begin{figure}[!htb]
    \centering
    \includegraphics[width=0.6\textwidth]{"Forks Prevented".pdf}
    \caption{Mean number of forks prevented in a blockchain of length 50 and a network of 114 miners, with 5\% confidence intervals}
    \label{fig-forks-prevented}
\end{figure}

% Make sections large again!
\titleformat*{\section}{\LARGE\bfseries}

\chapter{Conclusion \& Future Work} \label{ch-conclusion}

We presented a new signaling scheme (called Anchors) that gives early insight to miners about the true hashing power division in the network.
% With a successful proof of concept implementation we demonstrated that our basic premise of low latencies for anchors compared to blocks is true and that anchors resolve forks quickly thereby maintaining chain stability. 
We created a proof of concept implementation of Anchors in the reference Bitcoin client (Bitcoin Core).
After benchmarking unmodified Bitcoin against a variety of configurations of Bitcoin with Anchor support, we can conclude that Anchors are successful in making Bitcoin more robust, reducing fork resolution times and preventing forks altogether, without any substantial downsides to the system at large. 
% This is possible with minimal changes to the Bitcoin protocol, as shown by the proof-of-concept support of Anchors added to the Bitcoin reference implementation. 
% The addition of Anchors as implemented is also a soft fork (is it? I remember some controversy about this), which means that a smooth rollout is possible without disruption of services.

However, work on Anchors is far from complete. 
Currently, our implementation of Anchors does not consider rewards, which is a critical component if this proof-of-concept needs to be deployed in the real-world.
Adding rewards might mean that we deviate from the current Bitcoin block structure, in which case, we might want to look at methods that could work as a "soft-fork" into Bitcoin.
% means that miners have no incentive to use Anchors.
As new information about the exact topology of Bitcoin's overlay network becomes available, it could be integrated into the testbed to bring the emulation closer to the real world system.

% Further, this thesis shows experiments conducted across a variety of $a$ values as a testament of the flexibility of Anchors, but does not present any work on the calculation on an ideal value of $a$ which maximises fork resolution and fork prevention while simultaneously minimizing Anchor and Block propagation times. 
% Calculation of the ideal chain weight contribution of an Anchor (and possible half-life) also remains to be investigated.

We implemented Anchors in Bitcoin because it is the most widely deployed PoW blockchain, but in theory, Anchors could also be incorporated in blockchains using the GHOST protocol \cite{GHOST}, or Ethereum \cite{EthereumOriginal} and give similar benefits.
We could also experiment with implementing selfish-mining \cite{selfishmining} in Bitcoin Core and then testing whether Anchors have the potential to mitigate such attacks.


%----------------------------------------------------------------------------------------
%	THESIS CONTENT - APPENDICES
%----------------------------------------------------------------------------------------

% \appendix

%\include{Appendices/AppendixA}

%----------------------------------------------------------------------------------------
%	BIBLIOGRAPHY
%----------------------------------------------------------------------------------------

% \bibliography{References}
\printbibliography[heading=bibintoc]

% \fixme{Add URLs to every ref, so page count increases!}

%----------------------------------------------------------------------------------------

\end{document}
