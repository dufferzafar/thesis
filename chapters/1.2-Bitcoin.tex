%----------------------------------------------------------------------------------------

\newpage
\section{Bitcoin} \label{intro-bitcoin}

Bitcoin is a digital currency (also called crypto-currency) that was first proposed by the pseudonymous \footnote{It is unknown whether Satoshi is a person or a group of persons. Multiple people have tried to present "proof" that they're indeed Satoshi but there is still doubt.} "Satoshi Nakamoto" \cite{bitcoinOriginal}. 

Full node.

Form a Bitcoin network. The network is responsible for maintains 

%----------------------------------------------------------------------------------------

\newpage
\subsection{Proof-of-Work Consensus} \label{btc-pow}

Proof of Work is one of the mechanisms used by a Blockchain system to achieve consensus on the set of records (transactions in case of a Digital Ledger) that are considered valid. 

\begin{itemize}
    \item 1 cpu 1 vote
    \item Nakamoto consensus?
\end{itemize}

%----------------------------------------------------------------------------------------

\newpage
\subsection{Bitcoin Core} \label{btc-arch}

The reference implementation of Bitcoin, known as Bitcoin Core \cite{bitcoinGithub}, was first written in C++ by Staoshi Nakamoto and released to the public on 9 Januray, 2009 \cite{bitcoinCoreWiki}. The development was later picked up by a community of developers, lead by Wladimir J. van der Laan (as of 2019), who continue to maintain the code on GitHub (a site that hosts open-source code repositories). 

% Mention BIPs?

% We used the stable version (v0.16 released on 26-02-2018) of Bitcoin Core that was available to us before we started working on this project. For reference, the latest stable release of Bitcoin is v0.18 and was released on 18-05-2019. 

Most of Bitcoin Core is written in C++ and is available for all major operating systems like Windows, macOS and Linux \cite{bitcoinDownload}. 

Bitcoin Core includes four clients, three of which are command line-based: 

\begin{itemize}
    \item \textit{bitcoind} - also called "bitcoin daemon" - is the main application that runs in background as a Bitcoin full node. It also contains a wallet implementation which can fully verify payments.

    \item \textit{bitcoin-cli} - once the daemon is running in background, this application acts as an interface for controlling the daemon's behaviour. It works by sending appropriate RPC commands to the daemon.

    \item \textit{bitcoin-tx} - 

    \item \textit{bitcoin-qt} - this GUI based application acts as a frontend 
\end{itemize}

The source repository also includes functional and unit-tests for the Core source code.

We now give some details on the design of Bitcoin Core:

\subsubsection{Modes of Operation} \label{btc-modes}

The Bitcoin daemon can be run in three modes which differ in the blockchain they maintain and the consensus parameters they use:

\begin{itemize}
    \item \textbf{Mainnet} - This is the default mode of operation, where the daemon 
    \item \textbf{Testnet} - 
    \item \textbf{Regtest} - 
\end{itemize}

\subsubsection{Protocols} \label{btc-interfaces}

A running Bitcoin daemon has support for three protocols, each of which serves a distinct purpose.

\begin{itemize}
    \item \textbf{Bitcoin Protocol} - This is the protocol that a Bitcoin node uses to send or receive data from other nodes in the Bitcoin network. This custom networking protocol is built upon TCP.
    %  and uses various message types 

    \item \textbf{HTTP REST API} - 

    \item \textbf{RPC API} - The Remote Procedure Call (RPC) interface of a Bitcoin node can be used by other clients to send commands to it that control the behaviour of the node.
    This interface can be password protected so only authorised users can send commands.
\end{itemize}

\begin{figure}[!htb]
    \centering
    \includegraphics[width=\textwidth]{"Bitcoin Protocols".pdf}
    \caption[Protocol interfaces used by Bitcoin Core.]
    {
        Protocol interfaces used by Bitcoin Core. \\
        \footnotesize
        \textit{bitcoin-cli} can only use the RPC API to send commands, while a Python script can use either 
    }
    \label{fig-btc-arch}
\end{figure}
